\documentclass[spanish]{beamer}
\usepackage[utf8]{inputenc}
\usepackage[spanish, es-ucroman, es-noquoting]{babel}
\usepackage{tikz} % Requerido dibujar los circulos en las casillas de las matrices
\usepackage{amsmath}
\usepackage{qtree} % Requerido para dibujar árboles
\usepackage{textpos} % Provee entorno textblock
\usepackage{mathdots} % Provee macro \iddots
\usepackage{listings} % Provee entorno lstlisting
\usepackage[nomessages]{fp} % http://ctan.org/pkg/fp


%%%%%%%%%%%%%%%%%%%%%%%%%%%%%%%%%%%%%%%%%%%%%%%%%%%%%%%%%%%%%%%%%%%%%%%%%%%%%%%%
% Modo handout (comentar para versión presentación en pantalla/proyector)
% \usepackage{pgfpages}
% \pgfpagesuselayout{4 on 1}[a4paper, landscape, border shrink=5mm]
% \setbeamertemplate{background canvas}{
%     \tikz \draw (current page.north west) rectangle (current page.south east);
% }
%%%%%%%%%%%%%%%%%%%%%%%%%%%%%%%%%%%%%%%%%%%%%%%%%%%%%%%%%%%%%%%%%%%%%%%%%%%%%%%%

% Quitar controles de navegación
\usenavigationsymbolstemplate{}

% Numerar las transparencias
\setbeamertemplate{footline}[frame number]

% Formato del pseudocódigo
\lstset{
    basicstyle=\scriptsize\ttfamily,
    keywordstyle=\bfseries,
    showstringspaces=false,
    morekeywords={procedure, if, and, for, to, return}
}
\renewcommand{\ttdefault}{pcr}

% Macros para marcar las casillas que componen los caminos
% Parámetros: número, color, estilo
\newcommand*\circledcolor[3]{\tikz[baseline=(char.base)]{
    \node[shape=circle,draw,inner sep=1pt,color=#2, #3] (char) {\hphantom{00}};
    \node[inner sep=1pt] (char) {$#1$};
}}
\newcommand*\circled[1]{\circledcolor{#1}{black}{solid}}
\newcommand*\circledw[1]{\circledcolor{#1}{white}{solid}}
\newcommand*\circledd[1]{\circledcolor{#1}{black}{dashed}}

% Atajos
\newcommand{\matA}{A \in \mathbb{N}^{m \times n}}
\newcommand{\marcar}[2]{%
    \FPeval{\result}{clip(#2 - 1)}%
    \only<-\result>{%
        \circledw{#1}%
    }%
    \only<#2->{%
       \circled{#1}%
    }%
}



\title{Programación Dinámica}
\subtitle{
    Algoritmos y Estructuras de Datos III \\
    \vspace{2em}
    Prueba de Oposición \\
    Ayudantía de Segunda \\
    Área Algoritmos
}
\author{Leandro Lovisolo \\ \footnotesize{\texttt{<leandro@leandro.me>}}}
\date{8 de Octubre de 2014}
\institute{
    Departamento de Computación \\
    Facultad de Ciencias Exactas y Naturales \\
    Universidad de Buenos Aires
}

\begin{document}
    \begin{frame}
        \titlepage
    \end{frame}

    \begin{frame}
        \frametitle{Problema}
        \framesubtitle{Resolvamos un problema aplicando la técnica de programación dinámica}

        \begin{block}{Práctica 3, ejercicio 8}
            Sea $\matA$ una matriz de números naturales. Se desea obtener un camino que empiece en la casilla superior izquierda $(1, 1)$ y termine en la casilla inferior derecha $(m, n)$, tal que minimice la suma de los valores de las casillas por las que pasa. En cada casilla $(i, j)$ hay a lo sumo dos movimientos posibles: ir hacia abajo $(i + 1, j)$ o ir hacia la derecha $(i, j + 1)$.
        \end{block}

        \pause

        \begin{example}
            \begin{columns}
                \column{.5\textwidth}

                \only<-2>{
                    $$ 
                     A =
                    \begin{bmatrix}
                        \circledw{2} & 8            & 3            & 4 \\
                        \circledw{5} & 3            & 4            & 5 \\
                        \circledw{1} & \circledw{2} & \circledw{2} & \circledw{1} \\
                        3            & 4            & 6            & \circledw{5}
                    \end{bmatrix}
                    $$
                }

                \pause

                \only<3->{
                    $$ 
                    A =
                    \begin{bmatrix}
                        \circled{2} & 8           & 3           & 4 \\
                        \circled{5} & 3           & 4           & 5 \\
                        \circled{1} & \circled{2} & \circled{2} & \circled{1} \\
                        3           & 4           & 6           & \circled{5}
                    \end{bmatrix}
                    $$
                }

                \pause

                \column{.5\textwidth}
                Valor de la solución:
                $$2 + 5 + 1 + 2 + 2 + 1 + 5 = 18$$
            \end{columns}
        \end{example}
    \end{frame}
    
    \begin{frame}
        \frametitle{Fórmula Recursiva}
        \framesubtitle{Definamos recursivamente el valor de una solución óptima}

        \pause

        \begin{block}{Función de costo:}
            \begin{align*}
                c(A, i, j) = & \text{ costo mínimo de desplazarnos desde } (i, j) \\
                             & \text{ hasta } (m, n) \text{ en } \matA.
            \end{align*}
        \end{block}

        \pause

        \begin{block}{Queremos hallar:}
            $c(A, 1, 1)$. \pause Es decir, el costo mínimo de desplazarnos desde $(1, 1)$ hasta $(m, n)$ en $\matA$.
        \end{block}

    \end{frame}

    \begin{frame}
        \frametitle{Fórmula Recursiva}
        \framesubtitle{Definamos recursivamente el valor de una solución óptima}

        \begin{block}{Fórmula recursiva:}
            \vspace{0.5em}
            $
            c(A, i, j) =
            \left\{
                \begin{array}{ll}
                    \uncover<2->{
                        a_{m,n}                                           & \mbox{si } i = m \mbox{ y } j = n \\
                    }
                    \uncover<3->{
                        a_{i,j} + c(A, i + 1, j)                          & \mbox{si } i < m \mbox{ y } j = n \\
                    }
                    \uncover<4->{
                        a_{i,j} + c(A, i, j + 1)                          & \mbox{si } i = m \mbox{ y } j < n \\
                    }
                    \uncover<5->{
                        a_{i,j} + min \left\{
                            \begin{array}{l}
                                c(A, i + 1, j), \\
                                c(A, i, j + 1)
                            \end{array}
                            \right\}                                      & \mbox{si } i < m \mbox{ y } j < n
                    }
                \end{array}
            \right.
            $
        \end{block}

        \begin{minipage}[c][10em][c]{\textwidth}
            \begin{center}
                \only<2>{
                        $$ 
                        \begin{bmatrix}
                            \ddots & \vdots & \vdots \\
                            \cdots & 2      & 1 \\
                            \cdots & 6      & \circled{5}
                        \end{bmatrix}
                        $$

                        Caso base.
                }

                \only<3>{
                    $$ 
                    \begin{bmatrix}
                        \ddots & \vdots      & \vdots       & \iddots\\
                        \cdots & \circled{4} & \circledd{6} & \cdots
                    \end{bmatrix}
                    $$

                    Sólo podemos movernos hacia la derecha.
                }

                \only<4>{
                    $$ 
                    \begin{bmatrix}
                        \ddots  & \vdots       \\
                        \cdots  & \circled{5}  \\
                        \cdots  & \circledd{1} \\
                        \iddots & \vdots       \\
                    \end{bmatrix}
                    $$

                    Sólo podemos movernos hacia abajo.
                }

                \only<5>{
                    $$ 
                    \begin{bmatrix}
                        \ddots  & \vdots       & \vdots       & \iddots \\
                        \cdots  & \circled{3}  & \circledd{4} & \cdots  \\
                        \cdots  & \circledd{2} & 2            & \cdots  \\
                        \iddots & \vdots       & \vdots       & \ddots  \\
                    \end{bmatrix}
                    $$

                    Podemos movernos tanto hacia la derecha como hacia abajo.
                }
            \end{center}
        \end{minipage}
    \end{frame}


    \begin{frame}
        \frametitle{Condiciones para poder aplicar Programación Dinámica}

        \begin{center}
            ¿Puedo resolver este ejercicio con programación dinámica?

            \pause

            \vspace{1em}
            ¿Qué condiciones se tienen que cumplir?
        \end{center}

        \pause

        \vspace{1em}
        El problema debe exhibir:

        \begin{itemize}
            \item<4-> Solapamiento de subproblemas
            \item<5-> Subestructura óptima
        \end{itemize}
    \end{frame}

    \begin{frame}
        \frametitle{Solapamiento de Subproblemas}
        \framesubtitle{Veamos que algunos subproblemas se computan más de una vez}

        \begin{center}
            \uncover<1->{
                $c(A, 1, 1) =  a_{1,1} + min\{ c(A, 2, 1), c(A, 1, 2)\}$
            }

            \uncover<2->{
                $c(A, 2, 1) =  a_{2,1} + min\{ c(A, 2, 2), c(A, 3, 1)\}$
            }

            \uncover<3->{
                $c(A, 1, 2) =  a_{1,2} + min\{ c(A, 1, 3), c(A, 2, 2)\}$
            }
        \end{center}

        \uncover<4->{
            \vspace{1em}
            El subproblema $c(A, 2, 2)$ se computa más de una vez.
        }

        \begin{minipage}[c][8em][c]{\textwidth}
            \begin{center}
                \only<1>{
                        $
                        \begin{bmatrix}
                            \circled{2}  & \circledd{8} & \circledw{3} & \cdots \\
                            \circledd{5} & 3            & 4            & \cdots \\
                            \circledw{1} & 2            & 2            & \cdots \\
                            \vdots       & \vdots       & \vdots       & \ddots
                        \end{bmatrix}
                        $
                }

                \only<2>{
                        $
                        \begin{bmatrix}
                            \circledw{2} & \circled{8}  & \circledd{3} & \cdots \\
                            5            & \circledd{3} & 4            & \cdots \\
                            \circledw{1} & 2            & 2            & \cdots \\
                            \vdots       & \vdots       & \vdots       & \ddots
                        \end{bmatrix}
                        $
                }

                \only<3>{
                        $
                        \begin{bmatrix}
                            2            & 8            & \circledw{3} & \cdots \\
                            \circled{5}  & \circledd{3} & 4            & \cdots \\
                            \circledd{1} & 2            & 2            & \cdots \\
                            \vdots       & \vdots       & \vdots       & \ddots
                        \end{bmatrix}
                        $
                }

                \only<4->{
                        $
                        \begin{bmatrix}
                            2            & 8           & \circledw{3} & \cdots \\
                            5            & \circled{3} & 4            & \cdots \\
                            \circledw{1} & 2           & 2            & \cdots \\
                            \vdots       & \vdots      & \vdots       & \ddots
                        \end{bmatrix}
                        $
                }

            \end{center}
        \end{minipage}

        \uncover<5->{
            El ejercicio exhibe solapamiento de subproblemas. \hfill $\square$
        }
    \end{frame}

    \begin{frame}
        \frametitle{Subestructura Óptima}
        \framesubtitle{Veamos que las soluciones de los subproblemas también son soluciones óptimas}

        \begin{minipage}[t][10em][t]{\textwidth}
            \only<1-3>{
                Sea $p$ un camino de costo mínimo entre $(1, 1)$ y $(m, n)$.
            }

            \only<2-3>{
                \vspace{1em}
                Notación: \\
                $\Sigma p = \text{ la suma de las coordenadas de la matriz que componen } p$.
            }

            \only<3>{
                \vspace{1em}
                Entonces $p$ es un camino entre $(1, 1)$ y $(m, n)$ tal que $\Sigma p$ es mínima.
            }

            \only<4-7>{
                Descompongamos a $p$ en dos subcaminos: $p = p_1 \cup p_2$.
            }

            \only<7>{
                \vspace{1em}
                Entonces $\Sigma p = \Sigma p_1 + \Sigma p_2$.
            }

            \only<8>{
                Si el problema \textbf{no} exhibe subestructura óptima, entonces es posible que $p_2$ no sea un camino de costo mínimo entre $(3, 2)$ y $(m, n)$.
            }

            \only<9-10>{
                Supongamos que $p_2$ \textbf{no} es un camino de costo mínimo entre $(3, 2)$ y $(m, n)$. 
            }

            \only<10>{
                \vspace{1em}
                Entonces existe otro camino $p_2'$ entre $(3, 2)$ y $(m, n)$ tal que

                $$\Sigma p_2' < \Sigma p_2.$$
            }

            \only<11->{
                ¡Pero entonces es mentira que $p$ es un camino de costo mínimo!
            }

            \only<12->{
                \vspace{1em}
                Sea $p' = p_1 \cup p_2'$. \only<13->{Entonces:}
            }

            \only<13->{
                \begin{center}
                    $
                        \begin{array}{rcl}
                                          \Sigma p' & = & \Sigma p_1 + \Sigma p_2' \\
                            \uncover<14->{          & < & \Sigma p_1 + \Sigma p_2  \\}
                            \uncover<15->{          & = & \Sigma p}
                        \end{array}
                    $
                \end{center}
            }

            \only<16->{
                Luego $\Sigma p' < \Sigma p$. \only<17->{Absurdo, pues $p$ es un camino de costo mínimo.}
            }

            \only<18>{
                \vspace{1em}
                El ejercicio exhibe subestructura óptima. \hfill $\square$
            }            
        \end{minipage}

        \begin{minipage}[t][8em][t]{\textwidth}
            \begin{center}
                \only<1-4>{
                    $$ 
                    \begin{bmatrix}
                        \circled{2} & *           & *           & * \\
                        \circled{5} & *           & *           & * \\
                        \circled{1} & \circled{2} & \circled{2} & \circled{1} \\
                        *           & *           & *           & \circled{5}
                    \end{bmatrix}
                    $$

                    \only<1>{$p$}
                    \only<2-3>{$\Sigma p = 2 + 5 + 1 + 2 + 2 + 1 + 5 = 18$}%
                    \only<4>{$p = p_1 \cup p_2$}%
                }

                \only<5>{
                    $$ 
                    \begin{bmatrix}
                        \circled{2} & *            & *            & * \\
                        \circled{5} & *            & *            & * \\
                        \circled{1} & \circledw{2} & \circledw{2} & \circledw{1} \\
                        *           & *            & *            & \circledw{5}
                    \end{bmatrix}
                    $$

                    $p_1$
                }

                \only<6>{
                    $$ 
                    \begin{bmatrix}
                        \circledw{2} & *           & *           & * \\
                        \circledw{5} & *           & *           & * \\
                        \circledw{1} & \circled{2} & \circled{2} & \circled{1} \\
                        *            & *           & *           & \circled{5}
                    \end{bmatrix}
                    $$

                    $p_2$
                }

                \only<7>{
                    $$ 
                    \begin{bmatrix}
                        \circled{2} & *           & *           & * \\
                        \circled{5} & *           & *           & * \\
                        \circled{1} & \circled{2} & \circled{2} & \circled{1} \\
                        *           & *           & *           & \circled{5}
                    \end{bmatrix}
                    $$

                    $\Sigma p = \Sigma p_1 + \Sigma p_2$
                }

                \only<8-9>{
                    $$ 
                    \begin{bmatrix}
                        \circledw{2} & *           & *           & * \\
                        \circledw{5} & *           & *           & * \\
                        \circledw{1} & \circled{2} & \circled{2} & \circled{1} \\
                        *            & *           & *           & \circled{5}
                    \end{bmatrix}
                    $$

                    $p_2$
                }

                \only<10-11>{
                    $$ 
                    \begin{bmatrix}
                        \circledw{2} & *            & *            & * \\
                        \circledw{5} & *            & *            & * \\
                        \circledw{1} & \circledd{2} & \circledw{2} & \circledw{1} \\
                        *            & \circledd{*} & \circledd{*} & \circledd{5}
                    \end{bmatrix}
                    $$

                    $p_2'$
                }


                \only<12-17>{
                    $$ 
                    \begin{bmatrix}
                        \circled{2} & *            & *            & * \\
                        \circled{5} & *            & *            & * \\
                        \circled{1} & \circledd{2} & \circledw{2} & \circledw{1} \\
                        *           & \circledd{*} & \circledd{*} & \circledd{5}
                    \end{bmatrix}
                    $$

                    $p'$
                }
            \end{center}
        \end{minipage}
    \end{frame}

    \newsavebox{\algoritmoTopDown}
    \begin{lrbox}{\algoritmoTopDown}
        \begin{lstlisting}[gobble=12]
            dp = matriz[m][n] de enteros inicializada en -1
            procedure C(A, i, j):
              if dp[i][j] = -1:
                if i = m and j = n: dp[i][j] = A[m][n]
                if i < m and j = n: dp[i][j] = A[i][j] + C(A, i + 1, j)
                if i = m and j < n: dp[i][j] = A[i][j] + C(A, i, j + 1)
                if i < m and j < n: dp[i][j] = A[i][j] + min{C(A, i + 1, j),
                                                               C(A, i, j + 1)}
              return dp[i][j]
        \end{lstlisting}
    \end{lrbox}

    \begin{frame}
        \frametitle{Algoritmo Top Down}
        \framesubtitle{Resolvamos el problema recursivamente, resolviendo cada subproblema sólo una vez}

        \vspace{1em}
        Recordemos la fórmula recursiva:

        \vspace{1em}
        $
        c(A, i, j) =
        \left\{
            \begin{array}{ll}
                a_{m,n}                                           & \mbox{si } i = m \mbox{ y } j = n \\
                a_{i,j} + c(A, i + 1, j)                          & \mbox{si } i < m \mbox{ y } j = n \\
                a_{i,j} + c(A, i, j + 1)                          & \mbox{si } i = m \mbox{ y } j < n \\
                a_{i,j} + min \left\{
                    \begin{array}{l}
                        c(A, i + 1, j), \\
                        c(A, i, j + 1)
                    \end{array}
                    \right\}                                      & \mbox{si } i < m \mbox{ y } j < n
            \end{array}
        \right.
        $

        \pause

        \vspace{1em}
        El algoritmo implementa la fórmula directamente.

        \vspace{1em}
        \usebox{\algoritmoTopDown}
    \end{frame}

    \newsavebox{\algoritmoBottomUp}
    \begin{lrbox}{\algoritmoBottomUp}
        \begin{lstlisting}[gobble=12]
            procedure C(A):
              dp = matriz[m][n] de enteros
              for j = n to 1:
                for i = m to 1:
                  if i = m and j = n: dp[i][j] = A[i][j]
                  if i < m and j = n: dp[i][j] = A[i][j] + dp[i + 1][j]
                  if i = m and j < n: dp[i][j] = A[i][j] + dp[i][j + 1]
                  if i < m and j < n: dp[i][j] = A[i][j] + min{dp[i + 1][j],
                                                               dp[i][j + 1]}
              return dp[1][1]
        \end{lstlisting}
    \end{lrbox}

    \begin{frame}
        \frametitle{Algoritmo Bottom Up}
        \framesubtitle{Reescribamos la función anterior de forma iterativa}

        \vspace{0.5em}
        Recorremos la matriz de abajo a arriba y de derecha a izquierda, partiendo de $(m,n)$ y terminando en $(1, 1)$.

        \pause

        \vspace{1em}
        \usebox{\algoritmoBottomUp}

        \pause

        \vspace{-2.5em}
        \begin{columns}
            \column{.5\textwidth}
            \begin{center}
                $$A = 
                \begin{bmatrix}
                    \circledw{2} & \circledw{8} & \circledw{3} & \circledw{4} \\
                    \circledw{5} & \circledw{3} & \circledw{4} & \circledw{5} \\
                    \circledw{1} & \circledw{2} & \circledw{2} & \circledw{1} \\
                    \circledw{3} & \circledw{4} & \circledw{6} & \circledw{5}
                \end{bmatrix}
                $$
            \end{center}

            \column{.5\textwidth}
            \begin{center}
                $$dp = 
                \begin{bmatrix}
                    \uncover<19->{\circled {18}} & \uncover<18->{\circledw{21}} & \uncover<17->{\circledw{15}} & \uncover<16->{\circledw{15}} \\
                    \uncover<15->{\circledw{16}} & \uncover<14->{\circledw{13}} & \uncover<13->{\circledw{12}} & \uncover<12->{\circledw{11}} \\
                    \uncover<11->{\circledw{11}} & \uncover<10->{\circledw{10}} & \uncover<9-> {\circledw{8}}  & \uncover<8-> {\circledw{6}}  \\
                    \uncover<7-> {\circledw{18}} & \uncover<6-> {\circledw{15}} & \uncover<5-> {\circledw{11}} & \uncover<4-> {\circledw{5}}
                \end{bmatrix}
                $$
            \end{center}
        \end{columns}

        \vspace{1em}
        \uncover<20->{
            Valor de una solución óptima: $C(A) = 18$.
        }
    \end{frame}

    \begin{frame}
        \frametitle{Construcción de una Solución Óptima}
        \framesubtitle{Ya sabemos el costo del camino óptimo; ahora hallemos ése camino}

        Construimos un camino óptimo mirando la matriz $dp$ y siguiendo la lógica del algoritmo, partiendo de $(1, 1)$ hasta llegar a $(m, n)$.

        \uncover<2->{
            \vspace{1em}
            Nos paramos en $A_{1,1}$, marcamos esa coordenada, comparamos $dp_{2,1}$ y $dp_{1,2}$ y nos movemos a $A_{2,1}$ o $A_{1,2}$ según cuál de las respectivas posiciones en $dp$ tenga menor valor. \uncover<4->{Repetimos hasta llegar a $A_{m,n}$.}
        }

        \uncover<2->{
            \begin{columns}
                \column{.4\textwidth}
                $$
                A =
                \begin{bmatrix}
                    \marcar{2}{3} & 8             & 3             & 4 \\
                    \marcar{5}{4} & 3             & 4             & 5 \\
                    \marcar{1}{5} & \marcar{2}{6} & \marcar{2}{7} & \marcar{1}{8} \\
                    3             & 4             & 6             & \marcar{5}{9}
                \end{bmatrix}
                $$

                \column{.4\textwidth}
                $$
                dp =
                \begin{bmatrix}
                    \marcar{18}{3} & 21             & 15            & 15 \\
                    \marcar{16}{4} & 13             & 12            & 11 \\
                    \marcar{11}{5} & \marcar{10}{6} & \marcar{8}{7} & \marcar{6}{8} \\
                    18             & 15             & 11            & \marcar{5}{9}
                \end{bmatrix}
                $$
            \end{columns}
        }

        \uncover<10->{
            \vspace{1em}
            Un camino óptimo es entonces el conjunto de coordenadas que marcamos hasta llegar a $(m, n)$.
        }
    \end{frame}

    \begin{frame}
        \frametitle{Análisis de Complejidad}
        \framesubtitle{Comparemos nuestro algoritmo con la versión \textit{naïve}}

        \begin{block}{Aplicando programación dinámica} 
            Llenamos un arreglo de dimensión $m \times n$ con costo $O(1)$ por coordenada. \uncover<2->{Cota de complejidad total: $O(m \times n)$.}
        \end{block}

        \begin{columns}
            \column{.5\textwidth}

            \uncover<3->{
                \begin{block}{Versión \textit{naïve}}
                    Como el algoritmo top down, pero sin memoización.

                    \uncover<4->{
                        \vspace{1em}
                        Se produce un árbol (binario) de recursión de altura $m + n$. La cantidad máxima de nodos en un árbol de esta altura es $2^{m + n + 1} - 1$.
                    }

                    \uncover<5->{
                        \vspace{1em}
                        La complejidad total está entonces acotada por $O(2^{m + n})$.
                    }
                \end{block}
            }

            \column{.4\textwidth}
            \begin{center}
                \uncover<4->{
                    \tiny{
                        \Tree [.$c(A,1,1)$ [ .$c(A,2,1)$ [.$c(A,3,1)$ {\ldots} ] [.$c(A,2,2)$ {\ldots} ] ]
                                           [ .$c(A,1,2)$ [.$c(A,2,2)$ {\ldots} ] [.$c(A,1,3)$ {\ldots} ] ] ]
                    }

                    \vspace{1em}
                    \normalsize{Árbol de recursión del algoritmo naïve}
                }
            \end{center}
        \end{columns}
    \end{frame}
\end{document}