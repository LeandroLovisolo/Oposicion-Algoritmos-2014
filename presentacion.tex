\documentclass[spanish]{beamer}
\usepackage[utf8]{inputenc}
\usepackage[spanish, es-ucroman, es-noquoting]{babel}
\usepackage{tikz} % Requerido dibujar los circulos en las casillas de las matrices
\usepackage{amsmath}
\usepackage[noend]{algpseudocode}
\usepackage{qtree} % Requerido para dibujar árboles
\usepackage{textpos} % Provee entorno textblock
\usepackage{mathdots} % Provee macro \iddots

%%%%%%%%%%%%%%%%%%%%%%%%%%%%%%%%%%%%%%%%%%%%%%%%%%%%%%%%%%%%%%%%%%%%%%%%%%%%%%%%
% Modo handout (comentar para versión presentación en pantalla/proyector)
% \usepackage{pgfpages}
% \pgfpagesuselayout{4 on 1}[a4paper, landscape, border shrink=5mm]
% \setbeamertemplate{background canvas}{
%     \tikz \draw (current page.north west) rectangle (current page.south east);
% }
%%%%%%%%%%%%%%%%%%%%%%%%%%%%%%%%%%%%%%%%%%%%%%%%%%%%%%%%%%%%%%%%%%%%%%%%%%%%%%%%

% Quitar controles de navegación
\usenavigationsymbolstemplate{}

% Numerar las transparencias
\setbeamertemplate{footline}[frame number]

% Indentación del pseudocódigo
\algrenewcommand\algorithmicindent{1.0em}

% Macro para marcar las casillas que componen los caminos
% Parámetros: número, color, estilo
\newcommand*\circledcolor[3]{\tikz[baseline=(char.base)]{
    \node[shape=circle,draw,inner sep=1pt,color=#2, #3] (char) {\hphantom{00}};
    \node[inner sep=1pt] (char) {#1};
}}
\newcommand*\circled[1]{\circledcolor{#1}{black}{solid}}
\newcommand*\circledw[1]{\circledcolor{#1}{white}{solid}}
\newcommand*\circledd[1]{\circledcolor{#1}{black}{dashed}}

% Atajos
\newcommand{\matA}{A \in \mathbb{N}^{m \times n}}

\title{Programación Dinámica}
\subtitle{
    Algoritmos y Estructuras de Datos III \\
    \vspace{2em}
    Prueba de Oposición \\
    Ayudantía de Segunda \\
    Área Algoritmos
}
\author{Leandro Lovisolo \\ \footnotesize{\texttt{<leandro@leandro.me>}}}
\date{8 de Octubre de 2014}
\institute{
    Departamento de Computación \\
    Facultad de Ciencias Exactas y Naturales \\
    Universidad de Buenos Aires
}

\begin{document}
    \begin{frame}
        \titlepage
    \end{frame}

    \begin{frame}
        \frametitle{Problema}
        \framesubtitle{Resolvamos un problema aplicando la técnica de programación dinámica}

        \begin{block}{Recorriendo una matriz}
            Sea $\matA$ una matriz de números naturales. Se desea obtener un camino que empiece en la casilla superior izquierda $(1, 1)$ y termine en la casilla inferior derecha $(m, n)$, tal que minimice la suma de los valores de las casillas por las que pasa. En cada casilla $(i, j)$ hay a lo sumo dos movimientos posibles: ir hacia abajo $(i + 1, j)$ o ir hacia la derecha $(i, j + 1)$.
        \end{block}

        \pause

        \begin{example}
            \begin{columns}
                \column{.5\textwidth}

                \only<-2>{
                    $$ 
                     A =
                    \begin{bmatrix}
                        \circledw{2} & 8            & 3            & 4 \\
                        \circledw{5} & 3            & 4            & 5 \\
                        \circledw{1} & \circledw{2} & \circledw{2} & \circledw{1} \\
                        3            & 4            & 6            & \circledw{5}
                    \end{bmatrix}
                    $$
                }

                \pause

                \only<3->{
                    $$ 
                    A =
                    \begin{bmatrix}
                        \circled{2} & 8           & 3           & 4 \\
                        \circled{5} & 3           & 4           & 5 \\
                        \circled{1} & \circled{2} & \circled{2} & \circled{1} \\
                        3           & 4           & 6           & \circled{5}
                    \end{bmatrix}
                    $$
                }

                \pause

                \column{.5\textwidth}
                Valor de la solución:
                $$2 + 5 + 1 + 2 + 2 + 1 + 5 = 18$$
            \end{columns}
        \end{example}
    \end{frame}
    
    \begin{frame}
        \frametitle{Fórmula Recursiva}
        \framesubtitle{Definamos recursivamente el valor de una solución óptima}

        \pause

        \begin{block}{Función de costo:}
            \begin{align*}
                c(A, i, j) = & \text{ costo mínimo de desplazarnos desde } (i, j) \\
                             & \text{ hasta } (m, n) \text{ en } \matA.
            \end{align*}
        \end{block}

        \pause

        \begin{block}{Queremos hallar:}
            $c(A, 1, 1)$. \pause Es decir, el costo mínimo de desplazarnos desde $(1, 1)$ hasta $(m, n)$ en $\matA$.
        \end{block}

    \end{frame}

    \begin{frame}
        \frametitle{Fórmula Recursiva}
        \framesubtitle{Definamos recursivamente el valor de una solución óptima}

        \begin{block}{Fórmula recursiva:}
            \vspace{0.5em}
            $
            c(A, i, j) =
            \left\{
                \begin{array}{ll}
                    \uncover<2->{
                        a_{m,n}                                           & \mbox{si } i = m \mbox{ y } j = n \\
                    }
                    \uncover<3->{
                        a_{i,j} + c(A, i + 1, j)                          & \mbox{si } i < m \mbox{ y } j = n \\
                    }
                    \uncover<4->{
                        a_{i,j} + c(A, i, j + 1)                          & \mbox{si } i = m \mbox{ y } j < n \\
                    }
                    \uncover<5->{
                        a_{i,j} + min \left\{
                            \begin{array}{l}
                                c(A, i + 1, j), \\
                                c(A, i, j + 1)
                            \end{array}
                            \right\}                                      & \mbox{si } i < m \mbox{ y } j < n
                    }
                \end{array}
            \right.
            $
        \end{block}

        \begin{minipage}[c][10em][c]{\textwidth}
            \begin{center}
                \only<2-2>{
                        $$ 
                        \begin{bmatrix}
                            \ddots & \vdots & \vdots \\
                            \cdots & 2      & 1 \\
                            \cdots & 6      & \circled{5}
                        \end{bmatrix}
                        $$

                        Caso base.
                }

                \only<3-3>{
                    $$ 
                    \begin{bmatrix}
                        \ddots & \vdots      & \vdots       & \iddots\\
                        \cdots & \circled{4} & \circledd{6} & \cdots
                    \end{bmatrix}
                    $$

                    Sólo podemos movernos hacia la derecha.
                }

                \only<4-4>{
                    $$ 
                    \begin{bmatrix}
                        \ddots  & \vdots       \\
                        \cdots  & \circled{5}  \\
                        \cdots  & \circledd{1} \\
                        \iddots & \vdots       \\
                    \end{bmatrix}
                    $$

                    Sólo podemos movernos hacia abajo.
                }

                \only<5-5>{
                    $$ 
                    \begin{bmatrix}
                        \ddots  & \vdots       & \vdots       & \iddots \\
                        \cdots  & \circled{3}  & \circledd{4} & \cdots  \\
                        \cdots  & \circledd{2} & 2            & \cdots  \\
                        \iddots & \vdots       & \vdots       & \ddots  \\
                    \end{bmatrix}
                    $$

                    Podemos movernos tanto hacia la derecha como hacia abajo.
                }
            \end{center}
        \end{minipage}
    \end{frame}


    \begin{frame}
        \frametitle{Condiciones para poder aplicar Programación Dinámica}

        \begin{center}
            ¿Puedo resolver este ejercicio con programación dinámica?

            \pause

            \vspace{1em}
            ¿Qué condiciones se tienen que cumplir?
        \end{center}

        \pause

        \vspace{1em}
        El problema debe exhibir:

        \begin{itemize}
            \item<4-> Solapamiento de subproblemas
            \item<5-> Subestructura óptima
        \end{itemize}
    \end{frame}

    \begin{frame}
        \frametitle{Solapamiento de Subproblemas}
        \framesubtitle{Veamos que algunos subproblemas se computan más de una vez}

        \begin{center}
            \uncover<1->{
                $c(A, 1, 1) =  a_{1,1} + min\{ c(A, 2, 1), c(A, 1, 2)\}$
            }

            \uncover<2->{
                $c(A, 2, 1) =  a_{2,1} + min\{ c(A, 2, 2), c(A, 3, 1)\}$
            }

            \uncover<3->{
                $c(A, 1, 2) =  a_{1,2} + min\{ c(A, 1, 3), c(A, 2, 2)\}$
            }

            \uncover<4->{
                \vspace{1em}
                El subproblema $c(A, 2, 2)$ se computa más de una vez.
            }
        \end{center}

        \begin{minipage}[c][8em][c]{\textwidth}
            \begin{center}
                \only<1-1>{
                        $
                        \begin{bmatrix}
                            \circled{2}  & \circledd{8} & \circledw{3} & \cdots \\
                            \circledd{5} & 3            & 4            & \cdots \\
                            \circledw{1} & 2            & 2            & \cdots \\
                            \vdots       & \vdots       & \vdots       & \ddots
                        \end{bmatrix}
                        $
                }

                \only<2-2>{
                        $
                        \begin{bmatrix}
                            \circledw{2} & \circled{8}  & \circledd{3} & \cdots \\
                            5            & \circledd{3} & 4            & \cdots \\
                            \circledw{1} & 2            & 2            & \cdots \\
                            \vdots       & \vdots       & \vdots       & \ddots
                        \end{bmatrix}
                        $
                }

                \only<3-3>{
                        $
                        \begin{bmatrix}
                            2            & 8            & \circledw{3} & \cdots \\
                            \circled{5}  & \circledd{3} & 4            & \cdots \\
                            \circledd{1} & 2            & 2            & \cdots \\
                            \vdots       & \vdots       & \vdots       & \ddots
                        \end{bmatrix}
                        $
                }

                \only<4->{
                        $
                        \begin{bmatrix}
                            2            & 8           & \circledw{3} & \cdots \\
                            5            & \circled{3} & 4            & \cdots \\
                            \circledw{1} & 2           & 2            & \cdots \\
                            \vdots       & \vdots      & \vdots       & \ddots
                        \end{bmatrix}
                        $
                }

            \end{center}
        \end{minipage}

        \begin{center}    
            \uncover<5->{
                El ejercicio exhibe solapamiento de subproblemas.
            }
        \end{center}    
    \end{frame}

    \begin{frame}
        \frametitle{Subestructura Óptima}
        \framesubtitle{Veamos que las soluciones de los subproblemas también son soluciones óptimas}

        Sea $\Sigma p$ la suma de las casillas por las que pasa un camino $p$.

        \vspace{1em}

        Sea $p = \{(1, 1), \ldots, (i_k, j_k), (i_{k+1}, j_{k+1}), \ldots, (m, n)\}$ un camino con costo mínimo entre $(1,1)$ y $(m,n)$.

        \vspace{1em}

        Descompongamos $p$ en dos caminos $p_1 = \{(1, 1), \ldots, (i_k, j_k)\}$ y $p_2 = \{(i_{k+1}, j_{k+1}), \ldots, (m, n)\}$, de manera que $p = p_1 \cup p_2$.

        \vspace{1em}

        Supongamos que existe otro camino $p'_2$ entre $(i_{k+1}, j_{k+1})$ y $(m,n)$ tal que $\Sigma p'_2 < \Sigma p_2$, y sea $p' = p_1 \cup p'_2$.

        \vspace{1em}

        Entonces $\Sigma p' = \Sigma p_1 + \Sigma p'_2 < \Sigma p_1 + \Sigma p_2 = \Sigma p$.

        \vspace{1em}

        Absurdo, pues $p$ es un camino con costo mínimo.

        \vspace{1em}

        Luego, el problema exhibe \textbf{subestructura óptima}.

    \end{frame}

    \begin{frame}
        \frametitle{Función de Costo Top Down}
        \framesubtitle{Resolvamos el problema recursivamente, resolviendo cada subproblema sólo una vez}

        El problema exhibe solapamiento de subproblemas y subestructura óptima, o sea que podemos aplicar la técnica de programación dinámica.

        \vspace{1em}

        \begin{algorithmic}
            \State $dp \leftarrow arreglo[m][n]$ de enteros con valor inicial $-1$
            \Procedure{c}{$A, i, j$}
                \If{$i > m$ \textbf{or} $j > n$} \Comment{posición inválida}
                    \State \textbf{return} $\infty$
                \EndIf
                \If{$dp[i][j] = -1$} \Comment{se computa sólo una vez}
                    \If{$i = m$ \textbf{and} $j = n$} \Comment{caso base}
                        \State $dp[i][j] = A[i][j]$
                    \Else \Comment{caso general}
                        \State $dp[i][j] = A[i][j] + min\{ \textsc{c}(A, i + 1, j), \textsc{c}(A, i, j + 1) \}$
                    \EndIf
                \EndIf
                \State \textbf{return} $dp[i][j]$
            \EndProcedure
        \end{algorithmic}


    \end{frame}

    \begin{frame}
        \frametitle{Función de Costo Bottom Up}
        \framesubtitle{Reescribamos la función anterior de forma iterativa}

        Recorremos la matriz de abajo a arriba y de derecha a izquierda, partiendo de $(m,n)$ y terminando en $(1, 1)$.

        \vspace{1em}

        \small{
            \begin{algorithmic}
                \Procedure{c}{$A$}
                    \State $dp \leftarrow arreglo[m][n]$ de enteros
                    \For{$j \leftarrow n$ \textbf{to} $1$}
                        \For{$i \leftarrow m$ \textbf{to} $1$}
                            \If{$i < m$ \textbf{and} $j < n$} \Comment{caso general}
                                \State $dp[i][j] \leftarrow A[i][j] + min\{ dp[i+1][j], dp[i][j+1] \}$
                            \ElsIf{$i < m$ \textbf{and} $j = n$} \Comment{última columna}
                                \State $dp[i][j] \leftarrow A[i][j] + dp[i+1][j]$
                            \ElsIf{$i = m$ \textbf{and} $j < n$} \Comment{última fila}
                                \State $dp[i][j] \leftarrow A[i][j] + dp[i][j+1]$
                            \Else \mbox{ } $dp[i][j] \leftarrow A[i][j]$  \Comment{caso base}
                            \EndIf
                        \EndFor
                    \EndFor
                    \State \textbf{return} $dp[1][1]$
                \EndProcedure
            \end{algorithmic}
        }
    \end{frame}

    \begin{frame}
        \frametitle{Construcción de una Solución Óptima}
        \framesubtitle{Ya sabemos el costo del camino óptimo; ahora hallemos ése camino}

        Construimos el camino óptimo mirando la matriz $dp$ donde guardamos los valores del camino óptimo hacia $(m, n)$ desde cada coordenada $(i, j)$.

        \vspace{1em}

        Nos paramos en $dp_{1,1}$, marcamos esa coordenada y nos movemos a la casilla contigua cuyo valor sea menor (sólo hacia la derecha ó abajo.) Repetimos hasta llegar a $dp_{m,n}$.

        \begin{columns}
            \column{.4\textwidth}
            $$
            A =
            \begin{bmatrix}
                \circled{2} & 8           & 3           & 4 \\
                \circled{5} & 3           & 4           & 5 \\
                \circled{1} & \circled{2} & \circled{2} & \circled{1} \\
                3           & 4           & 6           & \circled{5}
            \end{bmatrix}
            $$

            \column{.4\textwidth}
            $$
            dp =
            \begin{bmatrix}
                \circled{18} & 21           & 15          & 15 \\
                \circled{16} & 13           & 12          & 11 \\
                \circled{11} & \circled{10} & \circled{8} & \circled{6} \\
                18           & 15           & 11          & \circled{5}
            \end{bmatrix}
            $$
        \end{columns}

        \vspace{1em}

        El camino óptimo es entonces el conjunto de coordenadas que marcamos hasta llegar a $(m, n)$.
    \end{frame}

    \begin{frame}
        \frametitle{Análisis de Complejidad}
        \framesubtitle{Comparemos nuestro algoritmo con la versión \textit{naïve}}

        \begin{block}{Aplicando programación dinámica}
            Llenamos un arreglo de dimensión $m \times n$ con costo $O(1)$ por coordenada. Cota de complejidad total: $O(m \times n)$.
        \end{block}

        \begin{block}{Versión \textit{naïve}}
            \begin{columns}
                \column{.55\textwidth}
                Se produce un árbol (binario) de recursión de altura $m + n$. La cantidad máxima de nodos en un árbol de esta altura es $2^{m + n + 1} - 1$.

                \vspace{1em}

                La complejidad total está entonces acotada por $O(2^{m + n + 1})$.

                \column{.35\textwidth}

                \small{
                    \Tree [.$(1,1)$ [ .$(2,1)$ [.$(3,1)$ {\ldots} ] [.$(2,2)$ {\ldots} ] ]
                                    [ .$(1,2)$ [.$(2,2)$ {\ldots} ] [.$(1,3)$ {\ldots} ] ] ]
                }
            \end{columns}
        \end{block}
    \end{frame}
\end{document}